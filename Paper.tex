\documentclass[12pt, letter]{article}
\newcommand{\doctitle}{%
A Socio-Evolutionary Perspective of Human Altruism and Cooperation: Do Mirror Neurons hold the key?}
\newcommand{\bigO}{\ensuremath{\mathcal{O}}}
\newcommand{\tab}{\hspace*{12em}}
\usepackage{graphicx}
\usepackage{float}
\usepackage{listings}
\usepackage{comment}
\usepackage{setspace}
\usepackage{fancyvrb}
\usepackage{booktabs}
\usepackage[usenames,dvipsnames]{color}
\usepackage[center]{caption}
\usepackage{algorithm}
\usepackage{changepage}
\usepackage{algpseudocode}
\usepackage[margin=1in]{geometry}
\usepackage[usenames,dvipsnames]{color}
\usepackage{hyperref}
\hypersetup{
  colorlinks,
  citecolor=Violet,
  linkcolor=Black,
  urlcolor=Blue}
\onehalfspacing  
\begin{document}
\title{\textbf{\doctitle}\\
\textsc{CS 8893: Culture and Cognition\\ Term Paper}
}
  \author {Arvind Krishnaa Jagannathan \\ GT ID: 902891874}
  \date{}
\maketitle
%\begin{adjustwidth}{1.5cm}{0pt}
Altruism, defined as selfless sacrifice for the goodwill of another being, when taken at face value seems to be in stark contrast to the concept of natural selection. The ``mindlessly altruistic'' bacteria \cite{warneken2009roots} that live on certain trees, and sacrifice their life generation after generation to improve the sap production in the tree, would seem like an aberration to the general theory of fitness selection among classical evolutionary biologists. Although several theories, such as kin selection and reciprocal altruism have been postulated, they are not sufficiently backed up by observed or scientific proof. In this paper, I hope to review some of the work done in integrating the notion of altruism within the evolutionary framework. Also, in light of the recent discovery of mirror neurons, I posit altruism as an evolutionary trait realized within the mirror neuron (MN) system. Altruism seems to be a natural extension of one's empathy towards another conspecific, which has been shown \cite{decety2006social} to be linked with the sophistication of the MN system. Mirror neurons not only give us a biological foothold to explaining altruism, but also provide compelling explanations for phenomena like the \emph{Bystander Effect}. 

In addition, I hope to explore the inherently human desire to participate in cooperative tasks as part of a large social group and link it to our basic altruistic tendencies. I will focus on how culture and social interactions help shape these abilities as we grow up in a social environment. Finally, the principal question that I hope to address in this paper is if altruism and cooperative task completion exhibited by humans are ``intentional'' and fundamentally unique in the animal kingdom.
%\end{adjustwidth}

\section*{Defining Altruism}
\emph{Altruism} is defined as selfless concern for the well-being of others, even if it is at their own expense. A 50-year old construction worker jumping onto the tracks of a New York subway to save an unconscious woman from being crushed by an oncoming train is a shining example of altruism. This woman was in no way related to the construction worker, and yet he was willing to risk his own life in protecting her, even though there were thousands of ``concerned'' onlookers who took no action. Such incidents are significant, not only because of the magnitude of ``courage'' (or stupidity) involved in carrying out an action which so few would even dare think of, but also since such incidents happen very rarely; even fewer cases are reported publicly.

For the purposes of establishing altruism as a concept within the laws of evolution, it is necessary to broaden scenarios which can be covered by the term ``altruism''. By this I mean that we should not overlook everyday occurrences of selfless concern for others, which cannot be strictly classified as altruism. For instance, picking up somebody's fallen notebook, helping a blind person cross the road or even loaning some money for charity, should be considered to be on the fringe of altruism. Although one does not necessarily ``sacrifice'' anything in such circumstances, there is no immediate benefit that one can derive from these situations. Were we to restrict analysis of human behavior to only those ``heroic'' situations where a soldier saves his entire unit by jumping on top of a live grenade, then we simply may not have enough units of data to analyze. Thus, for the purposes of this paper, my definition of altruism would be actions that exhibit selfless concern for others, at some cost to the altruist; however executing these actions does not present the performer with any \emph{immediate} or \emph{forseeable} benefits. These actions may or may not involve any kind of \emph{sacrifice} from the benefactor. Therefore, in this paper altruism and ``prosocial behavior'' are used as interchangeable terms.

\subsection*{Is there a stand-off with Natural Selection?}

Darwin knew that altruism was a problem for his theory of evolution and natural selection \cite{warneken2009roots}, in particular he was worried about eusocial insects such as termites and ants who regularly sacrificed for one another. If only those species of animals which develop traits enabling them to survive are selected to live beyond a few generations, how is it that such self-sacrificing species have managed to survive for several thousand years? Several explanations for this potential snag in the evolution theory have come up, the most popular being that of kin selection and reciprocal altruism.

Kin selection, also known as inclusive fitness, dictates that in addition to maximizing its own fitness function, if certain strategies favor the reproductive success of an organism's relatives, its off-springs, siblings and so on, then such strategies may be executed, in the absence of any other viable alternative. These strategies may be at a cost to the organism's own survival and reproduction, but by ensuring the survival of its kin, with which the altruist organism shares a common gene pool, it indirectly ensures that its own genes are passed on to the surviving organism's progeny. It is not known, however, if this process occurs with any level of intentionality in the sacrificing animal. However, this theory does not explain how one organism can still sacrifice for a genetically unrelated one (such as the bacteria for the tree).

Reciprocal altruism aims to address this one short coming of the kin selection theory. According to the theory of reciprocal altruism, individuals help others to the degree that they can anticipate being helped in return. Thus individuals can quickly cultivate a reputation of being helpful (or unfairly ``consuming'' help) in its ``society''. Although these theories sound very reasonable and logical, their origin can be traced from empirical studies and the need to explain existing phenomena, rather than from a scientifically constructed set of experiments. Although the theories could be interpreted in several ways, they make it clear that the notion of altruism fits into the scheme of natural selection. Organisms which are viewed as ``helpful'' are more likely to receive reciprocity benefits immediately; if they are not alive at the time, these benefits will usually be passed onto their nearest kin. The impact that an organism's helpful reputation can have on itself and its kin depends largely on how the society (in which the organism lives) values a possible act of altruism.

\subsection*{Does altruism exist at all?} 
It is not uncommon for scholars to question the existence of altruism. It is possible, scholars argue, that actions carried out by an organism seems to be altruistic from a third-party point of view, but that may not be an accurate reflection of the intention of the actor. For instance, consider once again the incident of the construction worker and the unconscious woman in the New York subway station. Although popular perception is that the construction worker ``saved'' the woman, it is quite possible that the construction worker wanted to commit suicide, and by (mis)fortune, he managed to save someone else's life. Worse yet, the construction worker may have saved the woman's life for the highly vain reason of being in the news of the day. Obviously no one can claim that their interpretation of such delicate issues are always right, and therefore the question of whether altruism truly exists, can never be definitely answered. However a more interesting area of inquiry would be into the cultural and social roots of altruism. This study is especially interesting with regards to humans. But before progressing with this study, it is necessary to quantify altruism depending on the commodities involved between the benefactor and the receiver. Warneken and Tomasello \cite{tomasello2008origins} characterize three main types of altruism, established within their economic framework:
\begin{enumerate}
\item \textit{With respect to goods}: Sharing of goods such as food or money is classified as an act of \textit{generosity}. Acts of sharing are also included in this categorization.
\item \textit{With respect to services}: Activities such as fetching out-of-reach objects, or to provide medical assistance to a person in pain.
\item \textit{Sharing information}: Sharing methods and techniques of tool use, share everyday occurrences (perhaps even gossip) with one another.
\end{enumerate}
A very obvious observation from the above three types of altruism is that each of them have a different cost associated for the performer as well as varying degrees of benefits received by the recipient. The subsequent sections describe a few case studies performed by eminent anthropologists and cognitive scientists in classifying several ``helpful'' acts into one of these three categories, with more focus on human subjects.

\section*{Phylogenetic and Ontogenetic Origins of Altruism}
Although it is a potentially significant evolutionary trait, the importance of culture and social interactions in molding altruism as a favorable characteristic cannot be discounted, especially in extremely social animals like humans and other higher primates. Especially with regards to human altruism, it is interesting to examine if altruism is hardwired into our existence right from birth or is it acquired as a result of social pressures. Experiments conducted by Tomasello et. al \cite{warneken2009roots} \cite{tomasello2008origins} with young children, from the ages of 12-18 months seems to indicate the former. These children were willing to assist adults in situations which demanded them to understand the goals and intentions of the adults and intervene in a helpful as opposed to an ``obstructive'' manner. In one of these tasks, an experimenter used clothespins to hang towels on a line; when he accidentally dropped a clothespin on the the floor and unsuccessfully reached for it, the child under observation walked towards the clothespin, picked it up and handed it over to him. Note however, that the adult made no explicit request to the child to get the clothespin. He simply stood there reaching in vain for the fallen object. It is intriguing to notice how the child is able to ``empathize'' with the experimenter; she is able to realize that the experimenter ``desires'' to get the clothespin, but due to spatial constraints (artificially created in this experimental setting) is unable to fulfill that desire; since he cannot satisfy his desire, the adult will develop a ``negative state of mind''. ``Since I am typically not used to seeing adults disappointed``, the child thinks, ``I need to do something to make the adult happy.'' This train of thought in the child's mind, I argue, is what leads her to help the adult complete his task. 

In another setting, an experimenter-child pair were playing a game of jigsaw. The experimenter would intentionally place a wrong piece in the wrong place, and then would sigh in pretend disappointment. It was noticed that in almost all situations, the child playing with the adult would usually correct the mistake by removing the wrong piece and placing the correct one in its spot. Slightly older children, say around 3-4 years of age, would actually explain to the adults why the previous move was wrong, and what he/she did to correct that move. Children sometimes explicitly show disappointment or anger at individuals (usually adults) who did not offer help to another adult requiring it. This was part of a similar setting as before, where the child was moved into an observatory position and replaced by a passive adult, who took no notice of the experimenter's predicament. In such scenarios, older children tend to verbally chide the passive adult, sometimes even plead to help the adult in need. Some experts may contradict the conclusions of this study, as they would be of the opinion that such behavior is inculcated as a result of positive reinforcement from their parents. However, I would argue against this point-of-view, simply because such children are too young to understand the notions of positive and negative enforcement. In another series of experiments, conducted as part of the same study by Tomasello et al., some of the children who helped the adult ``in distress'' were awarded with a prize (usually a piece of candy), while few others were not. On subsequent experiments, the children who were offered the candy assisted on fewer occasions when compared to the ones who were not. This shows that by providing an external benefit to an intrinsically satisfying task, such as helping out an adult, reduces its value; proving empirically that altruism is inherently present in children even before they understand the meaning of material possessions. I believe that once we begin to understand the worth of these material objects, from our interactions in the society, we begin to temper this indiscriminate altruistic impulse which we possess as children. Another evidence confirming this early emergence of altruism in children, is the brief work done by Callaghan and Tomasello \cite{tomasello2008origins}. They have found out that children who grew up in cultures where parents allowed their children to grow with much less teaching and intervention, also showed similar helpful tendencies towards others. Thus, I take the stand that evolution and phylogeny has a bigger impact on our altruistic traits than society and parental teaching.

\subsection*{Is Altruism hard-wired through the Mirror Neuron Infrastructure?}
It is fairly common in a nursery ward, or in general any room with a few infants, when one of them start crying most others join in. It has also been shown that some infants try to console another crying infant by putting their arm around their necks or by patting them on the back. Such incidences show that children (usually very young ones atleast), are unable to control their empathetic urges towards others, more so with other children of the same age. This, I strongly feel, indicates the involvement of mirror neuron system of the infants, which due to their relatively lack of sophistication as compared to adults, may not be able to discern themselves from the emotions felt by the crying baby. 

Mirror neurons are a particular class of visuomotor neurons, originally discovered in area F5 of the monkey premotor cortex, that discharge both when the monkey does a particular action and when it observes another individual (monkey or human) doing a similar action \cite{rizzolatti2004mirror}. Such a mechanism is present in humans as well, however is far more advanced as compared to monkeys. It seems possible that, in such a case, one would not be able to differentiate between the stimulus experienced by oneself and that experienced by another being which one observes. Ramachandran \cite{ramachandran2000mirror} argues that there is an auxiliary mechanism present in primates, which exists to prevent such a confusion.  However, were this system not fully developed, such as in infants, I am inclined to believe that it is entirely possible that the ``notion of self'' may not be established at all. Thus, infants cannot help but ``put themselves in  another's shoes''.

Another interesting observation from the same experiment was that the infants (otherwise calm) tend to cry on fewer occasions when an adult was in distress (the adult was crying for the sake of the experiment) than when an infant began to cry (perhaps due to hunger). This indicates that an infant is able to ``relate'' better with another infant than an adult. Similar results were provided by Decety and Jackson \cite{decety2006social}, in which they found a greater portion of the mirror neurons to be activated when individuals observe, on video, themselves (or individuals resembling themselves) performing a very familiar activity. I posit that when one infant observes another crying, a vast majority of the mirror neurons fire, which ``tricks'' their mind into thinking that they themselves are crying, and as a consequence produce tears. It is fairly obvious to me, based on such evidence, that altruism is realized in biology through the mirror neuron system. Empathy towards a fellow-being is unavoidable, but as indicated earlier, through social conditioning, we have managed to restrain these ``impulses''.

\subsection*{Empathy as a driving force for Altruism}
Empathy is a broad term, but I will limit its definition to strict boundaries to better serve my argument that empathy is a necessary precursor to altruism. I define empathy, along the lines of de Waal \cite{de2008putting} as the capacity to (a) be affected by and share the emotional state of another, (b) be able to assess the reasons for the other's state, from \emph{their} point of view and finally (c) be able to identify with the other, completely \emph{adopting} their perspective. I argue that every act of altruism is empathy-based, or in other words altruism cannot exist without one establishing an empathy-link to the individual that they want to help. This empathy based altruism may be intrinsically rewarding, since it offers an emotional stake for the actor in the recipient's well-being. This is especially true in the case of mothers helping out their children. 

Before I delve deeper into the connections between empathy and altruism, it is necessary to provide an insight on the origins of empathy. de Waal raises a major question regarding empathy; Does empathy channel altruism in the direction that evolutionary theory would predict? He argues that empathy evolved in animals as the main proximate mechanism for altruism, and this causes altruism to be ``dispensed'' as dictated by the predictions from kin selection and reciprocal altruism theories.

Although cognition is often critical to empathy, it seems secondary to the actual ``need'' to understand our fellow being. Selection pressures forced humans (and higher primates) to evolve rapid emotional connectedness during our evolution. This is much clear from the perspective of parental care. Infants even before they are able to communicate verbally, are able to signal their intentions for food or affection through crying or making noises. Most parents are able to quickly understand the intention behind every action of the child; cry when hungry, scream when they want to be played with, inactive when they want to sleep and so on. Avian or mammalian parents who were alert to their offspring's needs were most likely favored by selection over those that were indifferent towards their children. Thus, by evolution preferring ``good parents'' empathy as a trait seemed to be favorable. Once this context is established, empathy can be applied beyond the context of raising children. Empathy became involved in a lot of social relationships. For instance, macaque monkeys lick and clean the wounds of its fellow group monkeys injured during food gathering. Groups which show this kind of empathy to their kin, have been observed to be generally larger and usually more successful at dominating a territory than other species (such as lemurs)  which do not help one another. From the human perspective, I believe that the earliest humans, used altruistic deeds as a means to gain the acceptance of a society. For the neanderthals, everyday was a fight for survival as they had to endure constant threats. Being part of a group was obviously beneficial, especially so for a weaker being, since the dangers of being killed by a wild beast were significantly reduced. This realization that there is strength in numbers, is probably the first seeds of the altruistic trait in the human phylogeny. Hunter-gatherer groups which began to look after one another, soon grew in dominance and began to live longer. We used language as a means to ``teach'' our young ones that it is good to be empathetic, and thousands of years of selection coupled with our propensity for learning and culture has strongly instilled in this principle in humans.

\subsubsection*{Levels of empathy}
Clearly at the lowest level, empathy can simply be viewed as one party being affected by another's emotional or arousal state. This kind of emotional connectedness is common among most species, especially among humans. I believe that there are three levels of empathy, each involving specific cognitive abilities on the part of the observer, which then manifest as altruism. These are (a) Impulse empathy (b) Sympathy (c) Empathetic Perspective Taking.
 
Impulse empathy, which requires the lowest levels of cognition, is a reflex-like, quickly dissipated sense of emotion, triggered when observing someone receive a particular stimulus. The roomful of crying babies, or a flock of birds all taking off at once, when one of them is startled can be grouped into this category. No understanding is required of what caused the initial reaction, but it simply involves wholly adopting another's emotional state. Parents giving in to their child's temper tantrums is another example of impulse empathy. Emotional responses to display of emotion in others is also a commonplace occurrence in most animals. Sometimes such responses tend to over-amplified. A recent experiment demonstrated that mice perceiving other mice in pain intensify their own response to pain \cite{langford2006social}. It is not too much of a stretch to imagine such responses to extend into action, resulting in the birth of altruism. A very compelling case arguing for impulse empathy leading to altruism, is an experiment from Masserman et al \cite{masserman1964altruistic}. In their experiment, they found that rhesus monkeys refuse to pull a chain that delivers food to them, provided such an action triggers pain in a companion (which was usually delivered in the form of an electric shock by the experimenters). The monkeys ``selflessly'' gave up the opportunity of receiving food, so that another monkey remains unharmed. Although it is not clear from their experiments whether the monkey in charge of pulling the chain, refuses to do so because it ``knows'' that the other monkey will receive a shock, but it is apparent this is some form of instinctive response and does not involve knowledge.

Sympathy, I define, as impulse empathy augmented by an understanding of the emotional state of another being. Psychology defines sympathy as, ``an affective response that consists of feelings of sorrow or concern for a distressed or needy other (rather than sharing the emotion of the other)''. In the case of Impulse empathy, the altruistic deeds performed by the actor (such as the monkey at chain), may have an alterior motive. The monkey simply may not have pulled the chain because of self interest; the monkey wants to alleviate the pain it suffers as a result of watching its companion in pain (perhaps via the MN system). As stated earlier, impulse empathy simply needs a visual stimulus. However, sympathy is an actively cognitive process. Concern for others is different from merely ``feeling their pain'' since it relies on a separation between internally and externally generated emotions. Sympathy can most often be seen in cases of sporting events, particularly tennis and amateur boxing, where the winner of the match goes over to the loser and gently puts their arm around his or her shoulder. Sympathy has been observed only in higher primates, and few other mammals such as dogs. Dogs often display sympathy in a manner similar to humans, probably because of several generations of co-habitation. A dog shows concern for a master, by going over and placing its head on their lap.

Empathetic perspective-taking is probably what psychologists would refer to as ``true empathy''. This involves understanding the state of emotion of another as well as the ability to adopt to their point of view. Its also a cognitive process, which is dependent on imagination and mental state attribution, Perspective-taking by itself should not be classified as empathy; instead it needs to be taken in combination with emotional engagement. Experimental evidence from Hare et al \cite{hare2001chimpanzees}, show that apes but not monkeys show some level of perspective-taking in their social behavior. A major manifestation of perspective-taking is the process of targeted helping. This is apparent in apes, where a mother ape returns to a whimpering young one trapped while swinging from one tree to another. Thus not only does the mother ape show concern towards her child, but also actively assists in mitigating the child's discomfort. For any individual to move ``beyond'' feeling sensitive towards another being, it requires them to understand that this emotional shift is to be attributed to the other and not to self. In other words, for one to truly take someone else's perspective, they should be aware of the source of the change in their emotional state. Apart from apes, and humans, the animals which have the most striking accounts of consolation and targeted helping are dolphins and elephants.

\subsubsection*{Towards altruism}
Not all altruistic behavior requires empathy. For instance, it seems quite fallacious to say that the mindlessly altruistic bacteria sacrifices its life out of concern for the tree. However, a case can be made for the fact that most acts of ``intended'' altruism arise as a natural extension of empathy. Considering the chimpanzees for a moment, the mother's concern for her offspring, leads to the process of \emph{sharing information}. Obviously a social group which shares information among each other, even though this may be at a cost for the member sharing the knowledge, often possess a cognitive advantage over groups which do not. I argue that empathetic perspective taking is simply a fine-line away from altruism. Anecdotal evidence indicate that dolphins usually ignore the risk of being killed, by helping out sick companions from drowning. Similarly, elephants usually ignore the risk of being hunted and try to free other elephants caught in a trap. There are hundreds of such examples of a targeted helping behavior in non-human animals (in addition to the countless cases among humans).

From such examples, it seems clear to me that altruism can be viewed loosely as ``extended empathy''. I define an empathy threshold, a hypothetical quantity which if below a certain range, results in altruism towards a recipient. I posit that this range could be the cost endured by the actor in executing the altruistic deed. The empathy threshold can be established by a whole host of situational factors; the person in need of help, the ``degree'' of help needed, the expertise of the benefactor and so on. However, I feel the empathy threshold in humans are determined by additional factors, unique only to us. For instance, as indicated earlier, most acts of altruism, even if they do not involve immediate benefits, are usually performed anticipating some future reciprocity either to the performer or to his progeny. Hence, I posit that the one's tendency to provide ``help'' to a total stranger would also depend on factors such as the ``influence'' exerted by the recipient, their appearance, personality and the social strata to which they belong. This may indeed be a reflection of human society as a whole; we are probably the only ones to be able to establish social hierarchy based on concepts such as money and power. In an unscientific and informal study I conducted, 90\% of the participants agreed that they would be more altruistically favorable towards the President of the United States than a stranger ``John Doe''. Other similar studies had two actors, one dressed in a very formal business attire and the other in a shoddy t-shirt and jeans, pretend to collapse on the side of a busy road in London. Although for a few minutes, neither of them received any attention, but about 12 minutes into the experiment there were a couple of roadside vendors who began to help the sharply dressed actor. The other actor was not even noticed for about 45 minutes, and it was only later when he started feigning sounds of hurt that passers-by  actually noticed that he was in distress. Even then most of the commuters merely ignored the actor; he received assistance only after 90 minutes into the experiment where a police officer helped him to his feet. In a minor modification, two actors, one male and another female were dressed up in sharp suits and the experiment was repeated with them. Here the well-dressed female actor received almost instantaneous assistance whereas it again took 12-15 minutes for the male actor to receive any assistance. Such studies clearly show that gender also seems to be a uniquely human factor in ``rationing'' altruism. Obviously such factors would matter only when the receiver and the benefactor have no relation; kin selection principles would be applicable if any relationship exists. In the next section, I will be exploring this phenomena, called the \emph{Bystander Effect}, in more detail, and present a mirror neuron oriented explanation for it.

\subsubsection*{Bystander Effect}
The Bystander effect or Genovese syndrome is a social phenomenon that refers to cases where individuals do not offer any means of help in an emergency situation to the victim when other people are present, such as the two experiments detailed previously. The probability of help has often appeared to be inversely related to the number of bystanders; in other words, the greater the number of bystanders, the less likely it is that any one of them will help. The mere presence of other bystanders greatly decreases intervention. Classical explanations for this phenomena indicate that as the number of bystanders increases, any given bystander is less likely to notice the situation, interpret the incident as a problem suffered by a victim and will less likely assume responsibility for taking an action to relieve the victim. However it is interesting to note that even if one person notices the situation and goes ahead to help, there is usually a few more people who come forward in offering assistance. Again Bystander Effect is also influenced by the seriousness of the victim's condition. The more serious or ``life-threatening'' the ailment of the victim is, the more likely it is that he/she will receive assistance.

Several explanations exist for this effect, the most notable being diffusion of responsibility. This occurs when observers all assume that someone else is going to intervene and so each individual feels less responsible and refrains from doing anything. Sometimes observers assume that a more qualified person (than themselves), such as a paramedic or a policeman, need to attend to the emergency. Such observers have usually identified the victim and have realized that the situation is indeed an emergency. They are unable to complete the transition from empathy to altruism by executing the final step of taking action. I posit that the reason for this inability might be because of an interfering role played by mirror neurons in raising the empathy threshold significantly. In such situations, there are a large majority of bystanders who do not actually notice a victim in trouble. However even if one person notices, they are unable to take action because they ``see'' around them lots of others who seem to be carrying on with their day-to-day activities. Thus a portion of mirror neurons, triggering pain centers of our brain, which should typically fire when we observe someone else in distress, is suppressed because of social pressure. Instead mirror neurons controlling motor function, dictate that we ``move'' away from the incident without taking time to address it. In other words, our empathetic desires get \emph{outweighed} by our needs to ``reflect'' what a majority of the people around us are doing. However, even if one person goes to assist the victim, this weighing scheme gets skewed, and our altruistic nature takes over. This is purely conjecture, as it is not as yet possible to study the rate of firing in mirror neurons in a common everyday outdoor setting.

Bystander effect, seems to be observed primarily in humans. There seems to be very little research which can conclusively establish the existence of this effect in other non-human animals. The most notable example of the Bystander Effect is the case of Kitty Genovese. On March 13 1964 Genovese, 28 years old, was on her way back to her Queens, New York, apartment from work at 3am when she was stabbed to death by a serial rapist and murderer. According to newspaper accounts, the attack lasted for at least a half an hour during which time Genovese screamed and pleaded for help. The murderer attacked Genovese and stabbed her, then fled the scene after attracting the attention of a neighbor. The killer then returned ten minutes later and finished the assault. Newspaper reports after Genovese's death claimed that 38 witnesses watched the stabbings and failed to intervene or even contact the police until after the attacker fled and Genovese had died  \cite{gansberg196438}.

\subsection*{The uniqueness of Human Altruism}
Although instance of the Bystander effect, and regular doomsday reporting from news channels may lead us to believe that humans are probably the least altruistic primates. Also, some of the factors which we use to determine whether an individual is worthy of being assisted, does not show us in good light. However, I would strongly argue that not only are humans the \emph{most} altruistic of all animals, but probably the only being who is intentionally so. It must be understood that humans, as a species need to be considered against every other species. Exceptional acts of altruism in humans or other animals should be considered as outliers, and we must confine ourselves to the ``average'' behavior of each species.

Most experiments, some detailed previously in this paper, provide strong evidence that indicate that chimpanzees have the cognitive capacity to infer other people's goals - which will provide them with the necessary knowledge and pre-requisites to offer help. However, it is still unclear the motivational component behind their helping another conspecific. In other words, one cannot say with certainty whether mother apes console their children to elevate the child's emotional state or for actually making themselves feel better. Of course, a similar case could be made with humans - do we help out purely because we want to ameliorate the other person, or is it because we want to build a reputation as a prosocial person amongst others. I feel that there might not be much merit pursuing whether ``true'' altruism exists. A more tangible study would be to investigate the cognitive skills displayed by various animals, humans included, while they carry out an altruistic task. Another important aspect to be looked at is the level of intentionality involved while these acts are carried out. It is with regard to these two points, I feel that humans are more ``involved'' cognitively when helping out someone in distress and are also more fully aware of the present scenario and the consequences of their deeds.

Warneken and Tomasello \cite{warneken2009roots} carried out several instrument helping tasks for a series of experiments with chimpanzees (similar to the ones they carried out with little children). Three nursery reared chimpanzees were used in these experiments, alongwith their caregivers. These chimpanzees helped out the caregivers in out-of-reach tasks, i.e scenarios in which the caregiver would drop an object on purpose and pretend that they cant reach it, by picking up the object and handing it to the caregiver. However the same chimpanzees showed no reliable helping in tasks such as the misplaced jigsaw puzzle. Children were found to help adults in significantly difficult tasks, such as removing an obstacle from the path of an adult who was stuck (the adult themselves could not do this because their hands were occupied); chimpanzees did not. One possible explanation is that the more complex goals for the caregiver in such tasks were not very obvious to the chimpanzees. In principle, they were willing to help, but only do so in situations when they are able too comprehend the needs and desires of the recipient, which as it turns out is very limited for chimps. However, as Warneken and Tomasello hint, this result may not be indicative of the ``average'' altruistic tendencies of chimpanzees, because they are human-trained chimpanzees involved in tasks along with the people that helped raise them. Hence they repeated the experiments on chimpanzees found in the forests of Uganda, and obtained similar results. They found that the chimpanzees were willing to help even if it included a slightly higher cost, such as climbing up a box, to fetch the out-of-reach object. However, the wild chimps also failed to help in cases where their human-trained cousins too offered no help.

They conducted another series of tests, in which the benefactor was no longer human. This was because with humans, they could possibly manipulate otherwise uncontrollable outcomes (such as the recipients reaction at being helped). They put chimpanzees in a situation in which 	one chimpanzee was faced with the problem of a locked door on the path leading to a source of food. This door could only be unlocked by another chimpanzee and only then can the first chimpanzee have access to food. Results showed that the second chimpanzee pulled the chain in most cases to let the first one get the food. It must be noted that all the chimpanzees in this experiment were genetically unrelated. These results also undoubtedly show that chimpanzees were willing to help other chimpanzees in ``novel'' situations even without any immediate return-benefit. The experiment was repeated with chimpanzees who did not help in several cases (of the first trial, lets call this Chimp 1), had its place interchanged with the chimp (Chimp 2) who was the one hoping for help in all those cases. It is interesting to note that the Chimp 2 usually pulled the chain to allow Chimp 1 to get to the food, even though Chimp 1 never helped out Chimp 2. This reveals the fact that the chimpanzees are unable to judge the ``reputation'' of the being it tries to help. This is however, not the case with humans (as discussed earlier humans incorporate reputation along with many other factor to determine if one is ``worthy'' of receiving help!). It has been observed by Olson and Spelke \cite{olson2008foundations}, that children as young as 3 years of age, are able to discern between favorable and non-favorable recipients of assistance. Children tend to share more with those who have been helpful to or nice to them previously. Tomasello \cite{tomasello2008origins} discovered that children of a similar age, shared their toys with those children who were nice to others.

Humans are probably the only ones to inflict altruistic punishment \cite{fehr2003nature}. They base their arguments on the \emph{ultimatum game}, in which two subjects have to agree on the division of a fixed sum of money. Person A, the proposer, can make exactly one proposal of how to divide the money. Then Person B, the responder, can accept or reject the proposed division. In the case of rejection, both receive nothing, whereas in the case of acceptance, the proposal is implemented. A sizeable number of people from a wide variety of cultures, even when facing high monetary stakes are willing to punish others at a cost to themselves to prevent unfair outcomes or to sanction unfair behavior. Results from Fehr and Fischbacher's experiments showed that proposals giving the responder shares below 25\% of the available money are rejected with a very high probability. A ``truly selfish'' person would have settled for whatever he/she got. Thus rejections in the ultimatum game can be viewed as an altruistic act because people have a notion of a fair outcome. Thus the responder is willing to forego an amount of money (thereby incurring a personal cost) to punish the person offering this unfair deal. This concept of ``fairness'' or what is expressed in literature as ``strong reciprocity'' seems present only in humans.

The other side of the coin is altruistic rewarding. Ever since the period of the hunter-gatherer, individuals who contribute more towards a group are often respected more than the average member. Their contributions might be in the form of providing food for the group from hunting, warding off attacks from other groups and wild animals or by providing education to the young members of the group. Such individuals are often rewarded with high social positions and on occasions presented with material benefits. In highly social groups, the kin of such individuals are often treated with as much respect and are usually better looked after. It is significant to note that humans are the only species who think that sharing knowledge is an act of altruism. Consequently humans are the only species who are capable of sharing their knowledge to another animal. Among the three types of altruism mentioned earlier, we can conclude that the first two categories (sharing possessions and sharing services) is fairly universal in the animal kingdom. However the third category (sharing information) is unique to humans, and this I believe has shaped our ability to interact and participate cooperatively in large social settings.

Cooperative behaviors may be favored in evolution because they enhance the individual’s opportunities for mating and coalition building. This would be the case, for example, if sharing valuable information or incurring dangers in defense of the group were taken by others as an honest signal of the individual’s otherwise unobservable traits as a mate or political ally \cite{bowles2003origins}. Our tendency to share knowledge has been a cultural catalyst over our evolution, thereby enabling us to cognitively leapfrog ahead of our closest primate cousins. As a result of the benefits reaped through information sharing, this ability has exploded exponentially. Through modern technology, we are able to find out about an uprising in Sudan, or a war in North Korea merely minutes after the event has begun. Although newspapers and television are now chasing the misguided goal of sensationalism, their origins lie in the altruistic desire to spread awareness to the people during desperate times, like the Great Plague or the Second World War. Nowadays, we share so much information through social media sites like Facebook or Twitter. Mothers share cooking recipes through email; families share pictures of their newborn via Facebook, celebrities share personal experiences via Twitter and so on. Here the cost involved to the actor is the process of actually acquiring the information through years of practice, but they are willing to sacrifice this knowledge for the benefit of the reader. Class rooms and research groups are probably the purest example of sharing information for the public benefit. Scholars are willing to share years of their work to their student, only for the benefit of their students. Engineers spend many man-months of effort and companies spend millions of dollars every year to develop open source software, just for the ease of use for the general public. The only ``benefit'' they can possibly receive is that their product is well-regarded among its users. Comparing our willingness (to some extent out love) for sharing such valuable information is mind-boggling, especially when put in the context of the wild monkeys. Experimentally, records indicate that a rhesus monkey who knows which tree has the best fruit of the forest, misleads and often lies about this knowledge when questioned by another monkey (sometimes its own children). Moreover, I am inclined to think that joint learning does not exist among other species simply because no other phyla of animals are even capable of coming up with a commonly understandable representation of knowledge. I explore more about joint learning and cooperative task completion in the next section.

\section*{Altruism and the Origin of Human Cooperation}

The primary way in which altruism affects human cooperation I feel is through the establishment of social norms. Humans, I believe, are the only ones in the animal kingdom to be governed by concepts such as ethics and social norms. Here is where culture perhaps has a bigger impact than evolution. For the most part, cultures try to encourage helpfulness and cooperation through various kinds of rules: be nice, be cooperative, share your stuff, do not lie and so on. Almost every culture tries to ``promote'' altruism even though they do not realize that it is inherently embedded in most of us. Nonetheless, most of us feel that we are bounded by these rules and act with a level of ``decency'' towards another human. We realize, at some point in our early childhood, that we are the targets of the judgment of others, who use these social norms as standards. We try to influence society's impression about us (atleast a majority of us), by adhering as closely to these norms as we can. Most of us even spend our entire life's energy and effort cultivating and defending our ``public self''.

Certainly there exists some mechanism among other species, especially the higher primates, which discourage members from engaging in anti-social activities. Tomasello believes that there are two basic rules which establishes this: (1) physical retaliation against others who harm them or their children and (2) avoiding non-cooperators when choosing partners \cite{tomasello2008origins}. However, neither of them are as complicated or sophisticated as the establishment of norms in human society. Again Tomasello broadly describes social norms to be defined three primary characteristics: (1) a global recognition of their force/power, (2) a mutual recognition of their general applicability and (3) a third party who enforce these norms (such as parents, the police, a court, the government and so on). Even children sometimes act as the enforcer of such norms (recall the child who was scolding the adult who was not helping another adult requiring assistance in one of the experiments described earlier). The ability to follow these social norms also result, I think, because our our innate ability to empathize at a higher cognitive level than chimps or other higher mammals. Children constantly have a threat in their mind, believing that others will treat them, just as they had treated others. I believe that our ability, or rather our forced ability to respect social norms, create strong positive motives in humans (as opposed to apes) to cooperate with others in a group activity. We know fully well that if we do not cooperate, then it will most likely lead to our exclusion from the group. This may make it difficult for the excluded individual to achieve his/her goal, and therefore makes non-cooperation, non-desirable. Thus we humans are able to create a sense of social responsibility, that is, subordinate one's personal goals and desires to that of the group. We establish a notion, popularly described by Thomas Nagel as ``he is me'' \cite{nagel1979possibility}. We believe that since everyone in a group is equal, the collective group goal is more significant than any one single person's goal. I claim that mirror neurons play a part in establishing this ``The group is always right'' mentality. I think that once a person is part of a focused group, there mirror neuron systems are polarized towards the intentions of the group. The notion of self is dissolved; we begin to realize that the group itself is an individual, and this individual has nothing different from ourself, hence the group must be us. Hence there is a subtle reversal of positions here: we are no longer part of the group; it is the group that is part of us. I emphatically urge that only humans can go to this extent of 	``perspective taking'', which lays the platform for joint task accomplishment. I must make my stand on this subject clear at this point. I feel that altruism is inherently an evolutionary trait does not ``originate'' out of socialization. However through cultural and social rules, we begin to incorporate altruism, and as an extension cooperative behavior, as acceptable and desirable traits.

Some eminent scholars, including Tomasello, believe that altruism is a ``bit player'' \cite{tomasello2008origins} in the development of collaborative tendencies in humans. I would disagree strongly with his point of view. I believe that altruism is a very essential pre-requiste for human collaboration. I present Tomasello's experimental evidence itself as a case in point. His experiments recorded that kids who were inclined to selfishly hog public toys, were often ignored when the kids played a game such as hide and seek. Cooperation cannot be possible unless one cannot understand and appreciate their collaborators' goals. A selfish person would not venture into breakthrough research, because even if he is interested in the field, the cost required to translate this interest into tangible work is too high; in addition immediate benefits cannot be guaranteed (the research might go bust after 4-5 years of fruitless research). Thus my belief is simple: cooperation cannot occur without having respect for the common goal, and this is not possible without having an altruistic streak.

\subsection*{The Nature of Human Cooperation}

Feats achieved through human cooperation, across time, are really impressive and it is hard for one to even imagine instances of cooperation among other species to the scale that we humans have managed to achieve. The international symbol of love, the Taj Mahal, involved the coordinated efforts of close to 20000 construction workers from different parts of the Indian sub-continent. The first moon mission, involved years of research and information sharing among thousands of American scientists, astronauts and government agencies. Such feats in the animal kingdom are unheard of; sure there are impressive beaver dams, such as the 850 metres long dam at the Wood Buffalo National Park in Alberta, Canada, but it would be foolish to assume they are on the same level as say perhaps, the International Space Station. There are fundamentally two main differences in human cooperation and cooperation as exhibited by other species: (1) Humans are probably the only species to work together with complete strangers. We often work with people who are genetically unrelated to us. However, whatever little evidence there is of chimpanzee cooperation, seems to indicate that they only work together with their kin - usually siblings, spouses or offsprings. (2) Not all joint human acts need to have immediate consequences or benefits. Take for instance the work of a research group, which might be cutting edge but have no practical applications at the moment. We undertake certain tasks as a group just because it seems ``interesting''. However, chimpanzees and usually other animals which take part in social activities, have a definite immediate goal that needs to be achieved. Usually this goal would be something very rudimentary such as protection against a potential predator, gathering food for the day or finding a mate. I believe humans are the only ones to have a collective urge to ``gather knowledge'', which is a realization of the concept of directed altruism. My claim has some empirical backing from studies of Warneken et al \cite{warneken2007spontaneous}. The studies had the participation of 14 to 24-month old children and three human raised juvenile chimpanzees, who were presented with four collaborative activities: two instrumental tasks in which there was a concrete goal and two social games in which there was no concrete goal other than playing the game itself. The human adult part­ner was instructed to quit acting at some point in the tasks as a way of determining the subject's understanding of the adult's commitment to the joint activity. The chimpanzees showed no interest in the social games, basically declining to participate in them. In the problem­ solving tasks, in contrast, they synchronized their behavior rela­tively skillfully with that of the human, and were largely successful in attaining the desire result. However, when the human partner stopped participating, no chimpanzee ever made a communicative attempt to reengage the human - suggesting that they had not formed with the human a joint goal. In contrast, the human children col­laborated in the social games as well as the instrumental tasks. More significantly, when the adult stopped participating in the activity, most children tried to persuade the adult to rejoin the activity; this communication suggests that the children in contrast to the chimpanzee had formed a shared goal with the adult, and were 	intended to achieve it.

Eminent psychologist Christophe Boesch presents the ability of chimpanzees in the forests of Ivory Coast, to hunt red colobus monkeys as a complex collaborative activity \cite{boesch2005joint}, in which each chimpanzee aims to reach a shared goal. However Tomasello disagrees that this activity can be equated with human level collaboration \cite{tomasello2005understanding}, and I agree with him. As Boesch describes the hunting activity, there is one single chimpanzee, called the driver, which chases the prey in a given direction. Other chimpanzees, called blockers, climb the nearby trees in the chase path, preventing the potential prey from changing direction. Finally there is an ambusher, which jumps in front of the monkey making escape virtually impossible. This description almost gives the reader a sense of football-like coordination in action; blockers are like the defense, the ambusher is like a receiver and the driver is like the quarterback. However, I feel that the hunting scenario can be viewed simply as a fortuitous coming together of a bunch of selfish individuals. The driver, blockers and ambusher do not necessarily have to remain in sync throughout the hunt, in the same sense that an architect, engineer and workers need to be, during the process of constructing a structure. It is very likely that one chimpanzee while hungry, saw a colobus monkey and thought, ``Lunch!''. Once the chimp began giving chase to the monkey (hence inadvertently becoming the driver), garnered the attention of other chimpanzees, who then look to snag the monkey themselves (thereby becoming blockers). Finally the strongest chimpanzee in the contingent (or perhaps the luckiest) jumps in front of the fleeing monkey and thus becoming the ambusher. Thus this study has several interpretations, and it would be unwise to assume that chimpanzees can form shared goals and act in a coordinated way to achieve that goal.

Not only have we been able to achieve great feats due to our cognitive superiority, but because of our ability to work towards a common goal. Moreover, I believe that actively participating in large group activities, has in fact accelerated our growth as intellectual beings. I reckon that we as a race learn new life sustaining skills (such as building a fire, tool use, mathematics and so on) only when we are part of a social setting. Such skills are impossible for one to pick up on their own. For evidence just ask a struggling high school student, who is unable to cope with math without a tutor. In his \textsc{Ted} talk on \emph{The neurons that shaped civilization}, Ramachandran posits that mirror neurons made it possible for us humans to learn things in ``one step'', which would have otherwise taken several thousand years. For instance, through evolution a polar bear may develop a warm coat after a few generations, but if a human learns to kill a polar bear and cut out its skin for warmth, others who are a part of his social group observe this and will be able to imitate and quickly learn the skill fairly rapidly. MN system provides a framework that enables such imitation learning. A possible ``selfish'' reason as to why we engage ourselves in group activities may be reduce the cognitive load required of one person to complete the task. Tasks which can be typically completed by one person, such as navigating a small boat (which essentially involves steering the boat and looking at a map for direction), are often divided among two or more people so that no individual has too many responsibilities. Hutchins would describe collaborative efforts as \emph{distributed cognition} \cite{hutchins1995cognition}. I strongly believe that distribution of cognition to other agents can at best be one of the motivations for engaging in joint tasks. However, to actively participate in such tasks and remain entrenched until the end, definitely requires the establishment of shared goals and intentions.

Another critical aspect of cooperation is the ability of successful division of labor and the ability of one partner to understand the role and responsibilities of another. Clearly the example presented by Boesch cannot be considered to entail these principles, primarily because of the open ended nature of the study. However humans possess what is described as a bird's eye view \cite{nagel1979possibility} of the activity. When members taking part in an activity are forced to swap responsibilities, they take sometime to adapt, but eventually manage to exchange the roles and accomplish their new responsibilities with some level of expertise. However, chimpanzees are unable to reproduce this ability to swap or reverse roles \cite{carpenter2005role}. This goes to show that chimpanzees, unlike humans are able to understand actions from a first-person as well as third-person point of view (which we established through the process of empathetic perspective taking), but are unable to obtain an integrated view of the two.

An important entwined aspect of a collaborative effort, is the process of coordinating attention. One of the earliest activities that most humans accomplish in coordination with another are termed in literature as joint attentional activities. Joint attentional activities can be best described as an infinite loop of observing the other's activities. In simpler terms, it can be described as an alternate sequence of following the other interacting party's gaze, i.e., a child first observes what the adult is looking at or pointing to. The next step involves actively involving the adult into the activity, either by asking the adult for help with some task, or by the child propagating the activity by initiating the gaze (or the pointing) and actively trying to get the adult to follow it. For such an activity to occur, it is necessary for a common goal, or interest, to be established. Having a common goal makes sure that both the parties involved in the activity can focus their attention on things relevant to the common goal. Thus the unique nature of human collaborative activities can be succinctly described as possessing the following structure: joint goal with interdependent and interchangeable roles, and coordination achieved by establishing a shared goal, which is augmented by joint attention and empathetic perspective taking. The skills necessary to participate in such activities, I urge, originates from our altruistic nature.

\section*{Concluding Remarks}

Throughout this paper, I feel that I have made liberal use of the term altruism. But if my definition of altruism, as a cost-involving acts of selflessness which does not provide any immediate benefit to the altruist, is acceptable to the reader, then I am sure that my uneconomical use of the word should be excuplated. I have tried, as much as possible, to provide experimental evidence to most of my claims. However, there are some statements which may seem as leaps of faith, but my belief is that the central thesis and overarching themes of this paper are fundamentally sound.

Altruism and natural selection can co-exist peacefully within the realms of the evolutionary theory, because I believe altruism is desirable trait developed because of selection, and not inspite of it. The recently discovered mirror neuron system may provide biological proof of this claim. I believe that the mirror neuron system develops in humans (and other higher primates), the necessary motivations to help out another animal in distress.  Cultural interactions via reinforcement of norms and societal pressures help in one understanding the benefits of being altruistic, but are not necessarily the origin of altruism (which lies in phylogeny).

Human altruism is ``cognitively'' superior to those of other animals, such as chimpanzees, elephants or dolphins, because we are able to understand the cause and effects involved in executing an altruistic action. It is this intentionality which enables humans to interact in unique and often awe-inspiring ways (when compared to other members of the  animal kingdom) with others. Altruism is a pre-requisite for us to engage in social activities, along with the capacity to establish joint goals. The ability to establish shared goals cannot exist unless one is able to understand the perspective of the group, which is achieved through empathy. Sharing goals is a two step process: (a) realizing that the group has a goal and (b) understanding that this goal is more important that one's individual desires. If an individual is ``completely selfish'', then it is impossible for that individual to accomplish either of these steps. There may be many reasons to collaborate: desire to achieve a goal which cannot be achieved alone, learn new skills, simplify a task which can be completed by an individual but would require a lot of effort and so on. Whatever be the motivation, all collaborative activities require a clear understanding of every member's roles and responsibilities and coordinated actions and attention.
 
\bibliographystyle{unsrt}	
\bibliography{myrefs}
\end{document}
