\documentclass[12pt, letter]{article}
\newcommand{\doctitle}{%
A Socio-Evolutionary Perspective of Human Altruism and Cooperation: Do Mirror Neurons hold the key?}
\newcommand{\bigO}{\ensuremath{\mathcal{O}}}
\newcommand{\tab}{\hspace*{12em}}
\usepackage{graphicx}
\usepackage{float}
\usepackage{listings}
\usepackage{comment}
\usepackage{setspace}
\usepackage{fancyvrb}
\usepackage{booktabs}
\usepackage[usenames,dvipsnames]{color}
\usepackage[center]{caption}
\usepackage{algorithm}
\usepackage{changepage}
\usepackage{algpseudocode}
\usepackage[margin=1in]{geometry}
\usepackage[usenames,dvipsnames]{color}
\usepackage{hyperref}
\hypersetup{
  colorlinks,
  citecolor=Violet,
  linkcolor=Black,
  urlcolor=Blue}
\onehalfspacing  
\begin{document}
\title{\textbf{\doctitle}\\
\textsc{CS 8893: Culture and Cognition\\ Term Paper}
}
  \author {Arvind Krishnaa Jagannathan \\ GT ID: 902891874}
  \date{}
\maketitle
%\begin{adjustwidth}{1.5cm}{0pt}
Altruism, defined as selfless sacrifice for the goodwill of another being, when taken at face value seems to be in stark contrast to the concept of natural selection. The ``mindlessly altruistic'' bacteria \cite{warneken2009roots} that live on certain trees, and sacrifice their life generation after generation to improve the sap production in the tree, would seem like an aberration to the general theory of fitness selection among classical evolutionary biologists. Although several theories, such as kin selection and reciprocal altruism have been postulated, they are not sufficiently backed up by observed or scientific proof. In this paper, I hope to review some of the work done in integrating the notion of altruism within the evolutionary framework. Also, in light of the recent discovery of mirror neurons, I posit altruism as an evolutionary trait realized within the mirror neuron (MN) system. Altruism seems to be a natural extension of one's empathy towards another conspecific, which has been shown \cite{decety2006social} to be linked with the sophistication of the MN system. Mirror neurons not only give us a biological foothold to explaining altruism, but also provide compelling evidence for phenomena like the \emph{Bystander Effect} \cite{gansberg196438}. 

In addition, I hope to explore the inherently human desire to participate in cooperative tasks as part of a large social group and link it to our basic altruistic tendencies. I will focus on how culture and social interactions help shape these abilities as we grow up in a social environment. Finally, the principal question that I hope to address in this paper is if altruism and cooperative task completion exhibited by humans are ``intentional'' and fundamentally unique in the animal kingdom.
%\end{adjustwidth}

\section*{Defining Altruism}
\emph{Altruism} is defined as selfless concern for the well-being of others, even if it is at their own expense. A 50-year old construction worker jumping onto the tracks of a New York subway to save an unconscious woman from being crushed by an oncoming train is a shining example of altruism. This woman was in no way related to the construction worker, and yet he was willing to risk his own life in protecting her, even though there were thousands of ``concerned'' onlookers who took no action. Such incidents are significant, not only because of the magnitude of ``courage'' (or stupidity) involved in carrying out an action which so few would even dare think of, but also since such incidents happen very rarely; even fewer cases are reported publicly.

For the purposes of establishing altruism as a concept within the laws of evolution, it is necessary to broaden scenarios which can be covered by the term ``altruism''. By this I mean that we should not overlook everyday occurrences of selfless concern for others, which cannot be strictly classified as altruism. For instance, picking up somebody's fallen notebook, helping a blind person cross the road or even loaning some money for charity, should be considered to be on the fringe of altruism. Although one does not necessarily ``sacrifice'' anything in such circumstances, there is no immediate benefit that one can derive from these situations. Were we to restrict analysis of human behavior to only those ``heroic'' situations where a soldier saves his entire unit by jumping on top of a live grenade, then we simply may not have enough units of data to analyze. Thus, for the purposes of this paper, my definition of altruism would be actions that exhibit selfless concern for others; however executing these actions does not present the performer with any immediate or forseeable benefits. These actions may or may not involve a sacrifice from the benefactor.






\bibliographystyle{unsrt}	
\bibliography{myrefs}
\end{document}
